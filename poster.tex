% Latex RWTH Poster template (based on a0poster document class)
% by Pablo Reble, Georg Wassen and Kirstin Schubert
% RWTH Aachen University
%
%This file may be distributed and/or modified
%
%1. under the LaTeX Project Public License and/or
%2. under the GNU Public License. 
%
% [21.01.2013]	 Version 0.1: First version of the template

\documentclass[a0,portrait,final]{a0poster}

\usepackage[english]{babel}
\usepackage[latin1]{inputenc}

\usepackage{listings}
\lstset{basicstyle=\ttfamily, language={C}}     % Standards for Listings

\usepackage{amsmath,amsthm, amssymb, latexsym,tikz}
\usepackage{multicol}
\usetikzlibrary{arrows,shapes}
\usepackage{pgfplots}
\pgfplotsset{
  every axis/.append style={thick}
}

\pgfdeclarelayer{background}
\pgfdeclarelayer{foreground}
\pgfsetlayers{background,main,foreground}

% \myfig - replacement for \figure
% necessary, since in multicol-environment 
% \figure won't work

\newcommand{\myfig}[3][0]{
  \begin{center}
    \vspace{1.5cm}
    \includegraphics[width=#3\hsize, angle=#1]{#2}
    \nobreak\medskip
  \end{center}}

% \mycaption - replacement for \caption
% necessary, since in multicol-environment \figure and
% therefore \caption won't work

%  \setcounter{figure}{1}
%  \newcommand{\mycaption}[1]{
%    \vspace{0.5cm}
%    \begin{quote}
%      %{{\sc Figure} \arabic{figure}: #1}
%      {#1}
%    \end{quote}
%    \vspace{1cm}
%    \stepcounter{figure}
%  }
%	

%\newenvironment{pcolumn}[1]{
%\begin{minipage}{#1}
  %\begin{center}
%  }{
  %\end{center}
%  \end{minipage}
%  }

\newcommand{\emailulg}[0]{antoine@ulg.ac.be}
\definecolor{cesamcolor}{RGB}{46,69,137}
%\definecolor{cesamcolor}{RGB}{158,89,152}
\usepackage{lfbsposter}

\begin{document}
%
  \title{Title of an interesting Poster,\\ which can also go into the second line}
  \author{Foo Duck and Bar Bunny}
  \affiliations{affiliations}
  \sffamily
  %\tikzexternaldisable

    \begin{poster}
    \makeheader
    \makefooter
    \vspace{18cm}


    \begin{multicols}{2}
    \hspace{2cm}
    \begin{minipage}[r]{\pcolwidth}

    %%
    %% top left box
    %%
\begin{posterbox}{20cm}{Items}
    This is an example for a poster box

    \end{posterbox}
        
    %%
    %% center left box
    %%
    \begin{posterbox}{20cm}{TikZ picture}
     graphs/ figures/important stuffs should also be created with these four colors : \\ \\
in priority : \\
{\Large \color{green} green}\\
{\Large \color{red} red}\\
{\Large \color{darkblue} darkblue}\\

If you don't have enough colors :\\
{\Large \color{lightblue} lightblue}\\
{\Large \color{purple} purple}

     
      \end{posterbox}
      
      %%
      %% bottom left box
      %%
      \begin{posterbox}{20cm}{Pictures}
      blabla3
      \end{posterbox}

      \begin{posterbox}{10cm}{Pictures}
      blabla3
      \end{posterbox}
   \end{minipage}

    %% end left column; start next one

    \begin{minipage}{\pcolwidth}
    %%
    %% top right box
    %%
    \begin{posterbox}{30cm}{Figure and Text}
    blabla4
    \end{posterbox}
    
    %%
    %% middle right box
    %%
    \begin{posterbox}{27cm}{Code}
  blabla5
    \end{posterbox} 
    
     \boxwidth=10cm
   \begin{posterbox}{20cm}{Conclusions}
  blabla6
    \end{posterbox}
    \end{minipage}
    \end{multicols}
  \end{poster}

\end{document}

